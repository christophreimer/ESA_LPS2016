% load packages
\usepackage[utf8]{inputenc}
\usepackage{mathpazo}
\usepackage[sfdefault,light]{roboto}
\usepackage[T1]{fontenc}
\usepackage{amsmath} 
\usepackage{blindtext}
\usepackage{comment}
\usepackage{enumerate}
\usepackage{xcolor}
\usepackage{boolexpr}
\usepackage{multicol}
\usetikzlibrary{arrows, decorations.markings}

% define default font
\renewcommand{\familydefault}{\sfdefault}

% color definitions
\definecolor{tublue}{HTML}{3571A1}
\definecolor{tulightblue}{HTML}{589DD4}
\definecolor{tuorange}{HTML}{F6C07C}

% block style definitions
\defineblockstyle{myDefault}{titlewidthscale=1, bodywidthscale=1, titleleft, titleoffsetx=0cm, titleoffsety=0cm, bodyoffsetx=0cm, bodyoffsety=-1cm, bodyverticalshift=0pt, roundedcorners=0, linewidth=0.1cm, titleinnersep=0.5cm, bodyinnersep=0cm}
{\begin{scope}[line width=\blocklinewidth, rounded corners=\blockroundedcorners]
    \ifBlockHasTitle
        %\draw[draw=none]
        %    (blockbody.south west) rectangle (blocktitle.north east);
        \draw[color=tulightblue, loosely dotted, line width=1mm]
        (blocktitle.south west) -- (blocktitle.south east);
        %\draw[color=tublue, solid, line width=2mm]
        %([yshift=-2mm]blocktitle.south west) -- ([yshift=-2mm]blocktitle.south east);
    \else
        \draw[draw=none]
            (blockbody.south west) rectangle (blockbody.north east);
    \fi
\end{scope}}

% define title style
\definetitlestyle{myWave}{width=\paperwidth, roundedcorners=0, linewidth=0pt, innersep=1.5cm, titletotopverticalspace=0mm, titletoblockverticalspace=35mm, titlegraphictotitledistance=10pt, titletextscale=2}
{
    \coordinate (topleft) at (\titleposleft,\titlepostop);
    \coordinate (topright) at (\titleposright,\titlepostop);
    \coordinate (lefttoright) at (\titlewidth,0);
    \coordinate (head) at (0,\titlepostop-\titleposbottom);
    \coordinate (bottomright) at (\titleposright,0);
    \coordinate (footerheight) at (0, 9);
    %
    \draw[draw=none, left color=tublue, right color=tublue]%
        (topright) -- %
        (topleft) -- %
        ($(topleft) - (head)$) -- %
        ($(topright) - (head)$) -- cycle; %
    \draw[draw=none, left color=tuorange, right color=tuorange]%
        ($(topright) - (head)$ ) --  %
        ($(topleft) - (head)$) -- %
        ($(topleft) - (head)-(0,.5)$) -- %
        ($(topright) - (head)-(0,.5)$) -- cycle; %
    \draw[draw=none, left color=tulightblue, right color=tulightblue]%
        ($(topright) - (head)-(0,.5)$ ) --  %
        ($(topleft) - (head)-(0,.5)$) -- %
        ($(topleft) - (head)-(0,1.)$) -- %
        ($(topright) - (head)-(0,1.)$) -- cycle; %
    \draw[draw=none, left color=tublue, right color=tublue]%
        ($(topright) - (head)-(0,1.)$ ) --  %
        ($(topleft) - (head)-(0,1.)$) -- %
        ($(topleft) - (head)-(0,1.5)$) -- %
        ($(topright) - (head)-(0,1.5)$) -- cycle; %
    \draw[draw=none, left color=tublue, right color=tublue]%
        (bottomleft) --  %
        ($(bottomleft) + (\titlewidth, 0)$) -- %
        ($(bottomleft) + (\titlewidth, 0) + (footerheight)$) -- %
        ($(bottomleft) + (footerheight)$) -- cycle; %
    \draw[draw=none, left color=tuorange, right color=tuorange]%
        ($(bottomleft) + (footerheight)$) -- %
        ($(bottomleft) + (\titlewidth, 0) + (footerheight)$) -- %
        ($(bottomleft) + (\titlewidth, 0.5) + (footerheight)$) -- %
        ($(bottomleft) + (0, 0.5) + (footerheight)$) -- cycle; %
    \draw[draw=none, left color=tulightblue, right color=tulightblue]%
        ($(bottomleft) + (0, 0.5) + (footerheight)$) -- %
        ($(bottomleft) + (\titlewidth, 0.5) + (footerheight)$) -- %
        ($(bottomleft) + (\titlewidth, 1.) + (footerheight)$) -- %
        ($(bottomleft) + (0, 1.) + (footerheight)$) -- cycle; %
    \draw[draw=none, left color=tublue, right color=tublue]%
        ($(bottomleft) + (0, 1.) + (footerheight)$) -- %
        ($(bottomleft) + (\titlewidth, 1.) + (footerheight)$) -- %
        ($(bottomleft) + (\titlewidth, 1.5) + (footerheight)$) -- %
        ($(bottomleft) + (0, 1.5) + (footerheight)$) -- cycle; %
}

% define color palette
\definecolorpalette{sampleColorPalette} {
  \definecolor{colorOne}{named}{tublue}
  \definecolor{colorTwo}{named}{tuorange}
  \definecolor{colorThree}{named}{tulightblue}
}

% set color styles
\definecolorstyle{sampleColorStyle} {
  \definecolor{colorOne}{named}{blue}
  \definecolor{colorTwo}{named}{yellow}
  \definecolor{colorThree}{named}{orange}
  }{
  % Background Colors
  \colorlet{backgroundcolor}{white}
  \colorlet{framecolor}{white}
  % Title Colors
  \colorlet{titlefgcolor}{white}
  \colorlet{titlebgcolor}{tublue}
  % Block Colors
  \colorlet{blocktitlebgcolor}{white}
  \colorlet{blocktitlefgcolor}{tublue}
  \colorlet{blockbodybgcolor}{white}
  \colorlet{blockbodyfgcolor}{black}
  % Innerblock Colors
  \colorlet{innerblocktitlebgcolor}{white}
  \colorlet{innerblocktitlefgcolor}{black}
  \colorlet{innerblockbodybgcolor}{black}
  \colorlet{innerblockbodyfgcolor}{black}
  % Note colors
  \colorlet{notefgcolor}{black}
  \colorlet{notebgcolor}{colorTwo!50!white}
  \colorlet{noteframecolor}{colorTwo}
}

% define note style
\definenotestyle{myNotestyle}{targetoffsetx=0pt, targetoffsety=0pt, angle=45, radius=8cm, width=6cm, connection=true, rotate=0, roundedcorners=0, linewidth=1pt, innersep=0pt}
{
  \ifNoteHasConnection
      \draw[thick] (notecenter) -- (notetarget)
      node{$\bullet$};
  \fi
  \draw[draw=notebgcolor,fill=notebgcolor,rotate=\noterotate](notecenter.south west) rectangle (notecenter.north east);
}

% define final poster layout
\definelayouttheme{sample}{
  \usecolorstyle[colorPalette=sampleColorPalette]{sampleColorStyle}
  \usebackgroundstyle{sample}
  \usetitlestyle{myWave}
  \useblockstyle{myDefault}
  \useinnerblockstyle{Minimal}
  \usenotestyle{myNotestyle}
}

\usetheme{sample}

