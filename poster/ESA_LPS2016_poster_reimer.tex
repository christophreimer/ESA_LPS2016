\documentclass[20pt, a0paper, portrait, innermargin=3cm, colspace=2.5cm]{tikzposter}

% load packages
\usepackage[utf8]{inputenc}
\usepackage{mathpazo}
\usepackage[sfdefault,light]{roboto}
\usepackage[T1]{fontenc}
\usepackage{amsmath} 
\usepackage{blindtext}
\usepackage{comment}
\usepackage{enumerate}
\usepackage{xcolor}
\usepackage{boolexpr}
\usepackage{multicol}
\usetikzlibrary{arrows, decorations.markings}

% define default font
\renewcommand{\familydefault}{\sfdefault}

% define column separation
\setlength{\columnsep}{2cm}

% color definitions
\definecolor{tublue}{HTML}{3571A1}
\definecolor{tulightblue}{HTML}{589DD4}
\definecolor{tuorange}{HTML}{F6C07C}

% block style definitions
\defineblockstyle{myDefault}{titlewidthscale=1, bodywidthscale=1, titleleft, titleoffsetx=0pt, titleoffsety=0cm, bodyoffsetx=0cm, bodyoffsety=-1cm, bodyverticalshift=0pt, roundedcorners=0, linewidth=0.1cm, titleinnersep=.05cm, bodyinnersep=0cm}
{\begin{scope}[line width=\blocklinewidth, rounded corners=\blockroundedcorners]
    \ifBlockHasTitle
        %\draw[draw=none]
        %    (blockbody.south west) rectangle (blocktitle.north east);
        \draw[color=tuorange, solid, line width=2mm]
        (blocktitle.south west) -- (blocktitle.south east);
        %\draw[color=tublue, solid, line width=2mm]
        %([yshift=-2mm]blocktitle.south west) -- ([yshift=-2mm]blocktitle.south east);
    \else
        \draw[draw=none]
            (blockbody.south west) rectangle (blockbody.north east);
    \fi
\end{scope}}

% define title style
\definetitlestyle{myWave}{width=\paperwidth, roundedcorners=0, linewidth=0pt, innersep=1.5cm, titletotopverticalspace=0mm, titletoblockverticalspace=35mm, titlegraphictotitledistance=10pt, titletextscale=2}
{
    \coordinate (topleft) at (\titleposleft,\titlepostop);
    \coordinate (topright) at (\titleposright,\titlepostop);
    \coordinate (lefttoright) at (\titlewidth,0);
    \coordinate (head) at (0,\titlepostop-\titleposbottom);
    \coordinate (bottomright) at (\titleposright,0);
    %
    \draw[draw=none, left color=tublue, right color=tublue]%
        (topright) -- %
        (topleft) -- %
        ($(topleft) - (head)$) -- %
        ($(topright) - (head)$) -- cycle; %
    \draw[draw=none, left color=tuorange, right color=tuorange]%
        ($(topright) - (head)$ ) --  %
        ($(topleft) - (head)$) -- %
        ($(topleft) - (head)-(0,.5)$) -- %
        ($(topright) - (head)-(0,.5)$) -- cycle; %
    \draw[draw=none, left color=tulightblue, right color=tulightblue]%
        ($(topright) - (head)-(0,.5)$ ) --  %
        ($(topleft) - (head)-(0,.5)$) -- %
        ($(topleft) - (head)-(0,1.2)$) -- %
        ($(topright) - (head)-(0,1.2)$) -- cycle; %
    \draw[draw=none, left color=tublue, right color=tublue]%
        ($(topright) - (head)-(0,1.2)$ ) --  %
        ($(topleft) - (head)-(0,1.2)$) -- %
        ($(topleft) - (head)-(0,1.5)$) -- %
        ($(topright) - (head)-(0,1.5)$) -- cycle; %
    \draw[draw=none, left color=tublue, right color=tublue]%
        (bottomleft) --  %
        ($(bottomleft) + (\titlewidth, 0)$) -- %
        ($(bottomleft) + (\titlewidth, 8)$) -- %
        ($(bottomleft) + (0, 8)$) -- cycle; %
    \draw[draw=none, left color=tuorange, right color=tuorange]%
        ($(bottomleft) + (0, 8)$) -- %
        ($(bottomleft) + (\titlewidth, 8)$) -- %
        ($(bottomleft) + (\titlewidth, 8.5)$) -- %
        ($(bottomleft) + (0, 8.5)$) -- cycle; %
    \draw[draw=none, left color=tulightblue, right color=tulightblue]%
        ($(bottomleft) + (0, 8.5)$) -- %
        ($(bottomleft) + (\titlewidth, 8.5)$) -- %
        ($(bottomleft) + (\titlewidth, 9.2)$) -- %
        ($(bottomleft) + (0, 9.2)$) -- cycle; %
    \draw[draw=none, left color=tublue, right color=tulightblue]%
        ($(bottomleft) + (0, 9.2)$) -- %
        ($(bottomleft) + (\titlewidth, 9.2)$) -- %
        ($(bottomleft) + (\titlewidth, 9.5)$) -- %
        ($(bottomleft) + (0, 9.5)$) -- cycle; %
}

% define color palette
\definecolorpalette{sampleColorPalette} {
  \definecolor{colorOne}{named}{tublue}
  \definecolor{colorTwo}{named}{tuorange}
  \definecolor{colorThree}{named}{tulightblue}
}

% set color styles
\definecolorstyle{sampleColorStyle} {
  \definecolor{colorOne}{named}{blue}
  \definecolor{colorTwo}{named}{yellow}
  \definecolor{colorThree}{named}{orange}
  }{
  % Background Colors
  \colorlet{backgroundcolor}{white}
  \colorlet{framecolor}{white}
  % Title Colors
  \colorlet{titlefgcolor}{white}
  \colorlet{titlebgcolor}{tublue}
  % Block Colors
  \colorlet{blocktitlebgcolor}{white}
  \colorlet{blocktitlefgcolor}{tublue}
  \colorlet{blockbodybgcolor}{white}
  \colorlet{blockbodyfgcolor}{black}
  % Innerblock Colors
  \colorlet{innerblocktitlebgcolor}{white}
  \colorlet{innerblocktitlefgcolor}{black}
  \colorlet{innerblockbodybgcolor}{black}
  \colorlet{innerblockbodyfgcolor}{black}
  % Note colors
  \colorlet{notefgcolor}{black}
  \colorlet{notebgcolor}{colorTwo!50!white}
  \colorlet{noteframecolor}{colorTwo}
}

% define note style
\definenotestyle{myNotestyle}{targetoffsetx=0pt, targetoffsety=0pt, angle=45, radius=8cm, width=6cm, connection=true, rotate=0, roundedcorners=0, linewidth=1pt, innersep=0pt}
{
  \ifNoteHasConnection
      \draw[thick] (notecenter) -- (notetarget)
      node{$\bullet$};
  \fi
  \draw[draw=notebgcolor,fill=notebgcolor,rotate=\noterotate](notecenter.south west) rectangle (notecenter.north east);
}

% define final poster layout
\definelayouttheme{sample}{
  \usecolorstyle[colorPalette=sampleColorPalette]{sampleColorStyle}
  \usebackgroundstyle{sample}
  \usetitlestyle{myWave}
  \useblockstyle{myDefault}
  \useinnerblockstyle{Minimal}
  \usenotestyle{myNotestyle}
}

\usetheme{sample}



\title{An enhanced Soil Moisture product from \\ the ERS Scatterometers}
\author{\underline{Christoph Reimer}, Isabella Pfeil and Wolfgang Wagner}
\date{ESA Living Planet Symposium, 9-13 May 2016, Prague, Czech Republic}
\titlegraphic{\includegraphics[scale=1]{graphics/TULogo.png}}
\institute{Vienna University of Technology, Department of Geodesy and Geoinformation}

\settitle{
  \begin{center}
    \begin{minipage}[t]{0.95\paperwidth}
        \begin{minipage}{0.18\linewidth}
          \@titlegraphic
        \end{minipage}
        \hfil
        \begin{minipage}{0.6\linewidth}
          \centering
          \color{titlefgcolor}
          {\bfseries \sffamily \Large \@title \par}
          \vspace*{1em}
          {\LARGE \sffamily \@author \par}
          \vspace*{1em}
          {\Large \sffamily \@institute \par}
          \vspace*{1em}
          {\large \sffamily \@date}
        \end{minipage}
        \hfill
        \begin{minipage}{0.18\linewidth}
          \begin{flushright}
            \includegraphics[scale=0.62]{graphics/SCIRoCCoLogo}
          \end{flushright}
        \end{minipage}
    \end{minipage}
  \end{center}
}

\begin{document}
 
  \maketitle

  \node[draw=none, rectangle, minimum width = .6cm, align=left, inner sep = 1cm,
  text=white, text width = 37cm, anchor=west] at (-42,-55)
  {\begin{enumerate}[{[1]}] \color{white} \item W. Wagner, G. Lemoine, and H.
      Rott, A method for estimating soil moisture from\\ ERS scatterometer and
      soil data, Remote Sensing of Environment, vol. 70, no. 2, pp. 191–207,
      1999. \item W. Wagner, G. Lemoine, M. Borgeaud, and H. Rott, ``A study of
      vegetation cover effects on ERS scatterometer data,'' \\ IEEE Transactions
      on Geoscience and Remote Sensing, vol. 37, no. 2II, pp. 938–948,
      1999. \item M. Vreugdenhil, W. A. Dorigo, W. Wagner, R. A. M. de Jeu, S.
      Hahn, and M. J. E. van Marle, ``Analyzing the Vegetation Parameterization
      in the TU-Wien ASCAT Soil Moisture Retrieval,'' IEEE TGRS, pp. 1–19,
      2016. \end{enumerate}};

  \node[draw=none, minimum width = 6cm, text width = 25cm, align=justify, inner
  sep = 1cm, text=white, anchor=west] at (-2,-55) {\textbf{Acknowledgments}\\ The
    authors would like to acknowledge the ESA Scatterometer Competence Center (SCIRoCCo) for their funding support. \\ \underline{http://scirocco.serco.it}};

  \node[draw=none, minimum width = 6cm, right=.5, align=right, text=white, inner sep = 1cm]
  at (26,-55) {\textbf{Christoph Reimer}\\ Vienna University of
    Technology\\ Department of Geodesy and Geoinformation\\ E-mail:
    christoph.reimer@geo.tuwien.ac.at\\ Web: http://rs.geo.tuwien.ac.at};  

  \begin{columns}

    \column{.5} 

    \block{MOTIVATION}{
      The first global soil moisture product from the scatterometer (ESCAT) on-board the ERS-1/2 missions was derived in the year 2000 by making use of the TU-Wien soil moisture algorithm [1]. 
      The ERS missions, especially ERS-2, underwent a number of major mission events (Figure \ref{figures:mission_events}) affecting the final soil moisture retrievals.       
      In the framework of the ESA funded project SCIRoCCo, this data archive was reconsidered with the objective to create the most complete and consistent ESCAT soil moisture products, taking full advantage of state of the art Level 1 and Level 2 scatterometer processors. 
      \begin{center}
        \begin{tikzfigure}[ERS-2 major mission events. \label{figures:mission_events}] 
          \includegraphics[width=0.45\textwidth]{figures/ERS_mission_overview.png}
        \end{tikzfigure}
      \end{center}
    }

    \block{Algorithmic Improvements (WARP 5.0 vs WARP 5.6)}{
      \begin{itemize}
        \item{Implementation/Improvements of Error Model.}
        \item{Improvement of Spatial Resampling}
        \item{Calculation of Freeze/Thaw parameter and corresponding Surface State Flags (SSF).}
        \item{Sensor Intra- / Inter-calibration}
      \end{itemize}
      
    }

    \block{RADIOMETRIC STABILITY}{Calibration Results}

    \column{.5} 

    \block{Product Overview}{TS and orbit, spatial resolution}

    \block{Validation Results}{ISMN and ERAINT}

    \block{Access the Data}{
      
    }

  \end{columns}

\end{document}
